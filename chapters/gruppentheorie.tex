\chapter{Gruppentheorie}

Sei $M$ eine Menge, sowie $*$ eine Funktion, welche zwei Elemente auf $M$ auf ein weiteres Element auf $M$ abbildet. $(M,*)$ nennen wir eine {\em Halbgruppe (Semigroup)}, wenn gilt:
\begin{enumerate}
  \item $*$ ist geschlossen, also $\forall e_1, e_2 \in M: e_1 * e_2 \in M$
  \item $*$ ist assoziativ, also $x*(y*z)=(x*y)*z$
\end{enumerate}

$(M,*,e)$ heißt {\em Monoid}, wenn zusätzlich gilt:

\begin{enumerate}
\setcounter{enumi}{2}
  \item $e$ ist neutrales Element, also $x*e = x = e*x$
\end{enumerate}

Ein Beispiel für einen Monoid ist die Operation $+$ über den Natürlichen Zahlen: $(\mathbb{N},+,0)$.

$(M,*,e)$ heißt {\em Gruppe}, wenn zusätzlich gilt:

\begin{enumerate}
\setcounter{enumi}{3}
  \item Für alle Elemente $x$ existiert ein inverses Element $y$:
  $$\forall x \in M\ \exists y \in M: x*y=e$$
\end{enumerate}


Es kann nur ein neutrales Element geben: Angenommen, wir hätten zwei unterschiedliche neutrale Elemente $e_1$ und $e_2$. Setze in Regel 3 $x=e_1$:
$$ e_1 * e_2 = e_1 = e_2 * e_1 $$
Setze in Regel 3 $x=e_2$:
$$ e_2 * e_1 = e_2 = e_1 * e_2 $$
Ineinandereingesetzt gilt damit auch:
$$ e_2 = e_1 * e_2 = e_1 $$
Dies ist ein Widerspruch zur Annahme.


Sei $(M,+,0)$ eine Gruppe. $(U,+,0)$ ist eine {\em Untergruppe} (oder {\em Subgruppe}), wenn gilt:
\begin{enumerate}
  \item $U \subseteq M$
  \item $\forall x,y \in U: x+y \in U$
  \item $\forall x \in U \exists y \in U: x+y=0$ (es gibt zu jedem $x$ ein Inverses)
  \item $0 \in U$
\end{enumerate}

Eine Eigenschaft von Subgruppen: Sei $\abs{M}$ die Ordnung (Mächtigkeit) der Menge $M$ (also die Anzahl der Elemente). Für alle Untergruppen $U$ von $M$ gilt: $\abs{M}\ mod\ \abs{U} = 0$. Die Mächtigkeit der Hauptgruppe ist also immer teilbar durch die Mächtigkeit der Untergruppe.

Notation: $\mathbb{Z}/10\mathbb{Z}$ bezeichnet die Menge der Elemente zwischen 0 und ausschließlich 10, also enthält die Elemente $\set{0,1,2,3,4,5,6,7,8,9})$. $\mathbb{Z}/10\mathbb{Z}^*$ folgt derselben Vorschrift, schließt allerdings die Null aus -- dies wären alos die Elemente $\set{1,2,3,4,5,6,7,8,9})$.

Beispiel für eine Subgruppe: Sei $G = (\mathbb{Z}/10\mathbb{Z}, \oplus, 0)$ eine Gruppe. $\abs{G}$ ist somit 10. Eine Subgruppe wäre beispielsweise $U = \set{0,5}$.
