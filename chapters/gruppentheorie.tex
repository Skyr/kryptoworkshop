\chapter{Gruppentheorie}

Sei $M$ eine Menge, sowie $*$ eine Funktion, welche zwei Elemente auf $M$ auf ein weiteres Element auf $M$ abbildet. $(M,*)$ nennen wir eine {\em Halbgruppe (Semigroup)}, wenn gilt:
\begin{enumerate}
  \item $*$ ist geschlossen, also $\forall e_1, e_2 \in M: e_1 * e_2 \in M$
  \item $*$ ist assoziativ, also $x*(y*z)=(x*y)*z$
\end{enumerate}

$(M,*,e)$ heißt {\em Monoid}, wenn zusätzlich gilt:

\begin{enumerate}
\setcounter{enumi}{2}
  \item $e$ ist neutrales Element, also $x*e = x = e*x$
\end{enumerate}

Ein Beispiel für einen Monoid ist die Operation $+$ über den Natürlichen Zahlen: $(\mathbb{N},+,0)$.

$(M,*,e)$ heißt {\em Gruppe}, wenn zusätzlich gilt:

\begin{enumerate}
\setcounter{enumi}{3}
  \item Für alle Elemente $x$ existiert ein inverses Element $y$:
  $$\forall x \in M\ \exists y \in M: x*y=e$$
\end{enumerate}


Es kann nur ein neutrales Element geben: Angenommen, wir hätten zwei unterschiedliche neutrale Elemente $e_1$ und $e_2$. Setze in Regel 3 $x=e_1$:
$$ e_1 * e_2 = e_1 = e_2 * e_1 $$
Setze in Regel 3 $x=e_2$:
$$ e_2 * e_1 = e_2 = e_1 * e_2 $$
Ineinandereingesetzt gilt damit auch:
$$ e_2 = e_1 * e_2 = e_1 $$
Dies ist ein Widerspruch zur Annahme.


