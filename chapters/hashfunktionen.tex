\chapter{Hashfunktionen}

Hashfunktionen sind Einwegfunktionen -- aus der Ausgabe kann nicht auf die Eingabe geschlossen werden; in der Kryptographie wird außerdem gefordert, daß es praktisch nicht möglich sein darf, zu einem Hash eine zugehörige Eingabe zu konstruieren. Der Eingabewert der Hashfunktion ist von variabler Größe und wird auf eine feste Hashlänge abgebildet.

Angriffsarten auf Hashfunktionen:
\begin{itemize}
\item{Pre-Image Attack (Urbild-Angriff)} sucht zu einem gegebenen Hash ein passendes Urbild: Er bestimmt zu einem gegebenen $y=h(x)$ ($x$ ist natürlich unbekannt) ein $x'$, sodass gilt: $y=h(x')$.
\item{Collision Attack (Kollisionsattacke)} sucht zwei Werte, welche auf denselben Hashwert abbilden: Der Angriff liefert zwei Werte $x_1$ und $x_2$, sodass gilt: $h(x_1)=h(x_2)$.
\item{Length Extension Attack} konstruiert aus einem Hash und der Länge des Ursprungstext einen (neuen) Hashwert, welcher zu dem (unbekannten) Ursprungstext und einer (berechneten) Text-Extension paßt: Gegeben sei $y = h(x)$.(wobei $x$ unbekannt ist). Mit dem Angriff bestimmt man ein $x'$ und $y'$, für welche gilt: $ y' = h(x \doubleplus x')$ ($\doubleplus$ bedeutet die Konkatenation, also das Aneinanderhängen).
\end{itemize}

\section{Merkle-Damgård}

Basismodell für viele bekannte Hashfunktionen (MD5, SHA, etc.).

\begin{center}
\def\svgwidth{0.8\columnwidth}
\input{images/merkle-darmgard-1.pdf_tex}
\end{center}

$h_0$ ist ein für den Algorithmus spezifischer Startwert. $f$ ist die Mixingfunktion, $out$ die abschließende Ausgabefunktion. In der Regel ist $out$ die Identitätsabbildung. Notwendige Voraussetzung für $f$ ist, daß sie nicht linear ist.

Orginalkonstruktion: Padding des letzten Blocks: Binäres Auffüllen mit einer 1, verbleibender Platz mit Nullen. Anschließend wird ein weiterer Block generiert, welcher die Länge des Eingabetexts beinhaltet.

|...100000000|       length|

MD5 und SHA optimieren hier: Die Längenangabe ist 64 Bit lang; sollte für das Padding mehr als 64 Bit sein, so wird die Länge im letzten Block mit untergebracht und kein neuer Block generiert.

|...100length|

|.........100|0000000length|

Die 1 zu Beginn des Paddings ist wichtig. Angenommen, wir hätten eine Blockgröße von 16 Bit und die beiden Eingabestrings $x_1 = 1011$ und $x_2 = 10110000$. Gäbe es die führende 1 nicht, würden beide Eingabestrings auf denselben Hashwert abgebildet werden.

Wenn $out$ die Identitätsfunktion ist, ist diese Konstruktion anfällig für Length Extension:

|...100000000|    oldlength|...........|.......100|L+L'+1|

Als Startwert wird der bisherige Hashwert benutzt. Der neu bestimmte Hashwert entspricht dem Hash, den man erhalten würde, wenn man die ursprünglichen Daten, deren Padding sowie die angehängten Daten bestimmen würde. Für die korrekte Bestimmung des abschließenden Paddings muß die Länge der ursprünglichen Daten bekannt sein.

