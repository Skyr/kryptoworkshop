\chapter{Klassische Chiffren}

\cite{Schneier1995}

\section{Caesar-Chiffre}

\section{rot13}

\section{rot n}

Monoalphabetische Chiffre

Unterscheidung Chiffre / Codierung: Chiffre besitzt Geheimnis; Codierung ist nur Darstellungsart.

\section{Modulo-Arithmetik}

\section{Angriff auf rot n}

Häufigkeitsanalyse

\section{Vigenère-Chiffre}

Polyalphabetische Chiffre

\section{Angriff auf Vigenere-Chiffre}

Zunächst Blocklänge bestimmen, anschließend für jede Gruppe eine separate Häufigkeitsanalye durchführen.

Blocklänge bestimmen: Durchprobieren, bis n-te Buchstaben charakteristische Verteilung ergeben

\section{Permutationen}

\section{Angriff auf Permutationen}

Blocklänge bestimmen: Markovketten-Modell der Sprache (Wahrscheinlichkeit, mit welcher ein Buchstabe auf den anderen folgt)
Durchprobieren, bis n-te Buchstaben dem Markov-Modell entsprechen

\section{Passwortstärke}

Längeres Passwort vs. größeres Alphabet: Länger ist besser

