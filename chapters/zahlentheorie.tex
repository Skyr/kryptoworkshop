\chapter{Logik}

Notation:

\begin{prooftree}
\AxiomC{$a \wedge b$}
\AxiomC{$a$}
\BinaryInfC{$b$}
\end{prooftree}

„Wenn alle Bedingungen oberhalb des Strichs erfüllt sind, gilt auch die Aussage unterhalb des Strichs“.


\section{Definition der natürlichen Zahlen}

„Die natürlichen Zahlen sind alle Werte von $x$, für die $n(x)$ gilt”:

$$\mathbb{N} = \left\{ x | n(x) \right\}$$

Notations-Definition: $x'$ ist der Nachfolger von $x$.

Definition der Addition:

$$x+0=x$$
$$x+y'=x'+y$$

Daraus läßt sich das Assoziativgesetz ableiten:

$$(a+b)+c = a+(b+c)$$

Definition der Multiplikation:

$$x \cdot 0=0$$
$$x \cdot y'=x+(x \cdot y)$$

Konvention: Es gilt Punkt vor Strich; zwischen Variablennamen kann das Multiplikationszeichen weggelassen werden.

\section{Definition der ganzen Zahlen}

$$\mathbb{Z}' = \left\{ (a,b) | a \in \mathbb{N}, b \in \mathbb{N}, a+x=b \right\}$$

$\mathbb{Z}'$ definiert noch nicht die ganzen Zahlen, sondern eine Menge von Gleichungen; über $x$ wird noch nichts ausgesagt -- für die Definition steht ja nur $\mathbb{N}$ zur Verfügung. Bildung von Äquivalenzklassen der Paare $(a,b)$:

$$ \exists k \in \mathbb{N} \big((a_1+k,b_1+k)=(a_2,b_2)\big) \vee \big((a_1,b_1)=(a_2+k,b_2+k)\big) \iff (a_1,b_1) \equiv (a_2,b_2) $$

Definition der ganzen Zahlen über „kanonische Repräsentanten“:

$$\mathbb{Z} = \left\{ (a,0) | a \in \mathbb{N} \right\} \cup \left\{ (0,b) | b \in \mathbb{N} \right\} $$

