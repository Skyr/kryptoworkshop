\chapter{Zahlentheorie}

Notation:

\begin{prooftree}
\AxiomC{$a \wedge b$}
\AxiomC{$a$}
\BinaryInfC{$b$}
\end{prooftree}

„Wenn alle Bedingungen oberhalb des Strichs erfüllt sind, gilt auch die Aussage unterhalb des Strichs“.


\section{Definition der natürlichen Zahlen}

\subsection{Axiome der natürlichen Zahlen}

„Die natürlichen Zahlen sind alle Werte von $x$, für die $n(x)$ gilt”:

$$\mathbb{N} = \left\{ x | n(x) \right\}$$

Notations-Definition: $x'$ ist der Nachfolger von $x$.


Die fünf Axiome der natürlichen Zahlen:

\begin{enumerate}
  \item $0$ ist eine natürliche Zahl:
  \begin{prooftree}
  \AxiomC{}
  \UnaryInfC{$n(0)$}
  \end{prooftree}

  \item Wenn $x$ eine natürliche Zahl ist, ist $x'$, der Nachfolger von $x$, auch eine natürliche Zahl:
  \begin{prooftree}
  \AxiomC{$n(x)$}
  \UnaryInfC{$n(x')$}
  \end{prooftree}

  \item Wenn der Nachfolger von $x$ gleich dem Nachfolger von $y$ ist, so ist $x$ gleich $y$:
  \begin{prooftree}
  \AxiomC{$x'=y'$}
  \UnaryInfC{$x=y$}
  \end{prooftree}

  \item Wenn $x$ eine natürliche Zahl ist, so ist ihr Nachfolger nie $0$:
  \begin{prooftree}
  \AxiomC{$n(x)$}
  \UnaryInfC{$x' \ne 0$}
  \end{prooftree}

  \item Wenn $x$ eine natürliche Zahl ist, so ist sie entweder $0$ oder sie besitzt einen Vorgänger $y$, welcher ebenfalls eine natürliche Zahl ist:
  \begin{prooftree}
  \AxiomC{$n(x)$}
  \UnaryInfC{$x=0 \vee (\exists y: y'=x \wedge n(y) )$}
  \end{prooftree}
\end{enumerate}


\subsection{Definition der Addition}

\begin{enumerate}
  \item $0$ ist das neutrale Element der Addition: $x+0=x$
  \item $x+y'=x'+y$ oder $x+y' = (x+y)'$
\end{enumerate}

Daraus läßt sich das Assoziativgesetz ableiten:

$$(a+b)+c = a+(b+c)$$


\subsection{Definition der Multiplikation}

\begin{enumerate}
  \item $x \cdot 0=0$
  \item $x \cdot y'=x+(x \cdot y)$
\end{enumerate}

Daraus läßt sich ebenfalls das Assoziativgesetz ableiten:

$$(a \cdot b) \cdot c = a \cdot (b \cdot c)$$

Konvention: Es gilt Punkt vor Strich; zwischen Variablennamen kann das Multiplikationszeichen weggelassen werden.


\section{Definition der ganzen Zahlen}

$$\mathbb{Z}' = \left\{ (a,b) | a \in \mathbb{N}, b \in \mathbb{N}, a+x=b \right\}$$

$\mathbb{Z}'$ definiert noch nicht die ganzen Zahlen, sondern eine Menge von Gleichungen; über $x$ wird noch nichts ausgesagt -- für die Definition steht ja nur $\mathbb{N}$ zur Verfügung. Bildung von Äquivalenzklassen der Paare $(a,b)$:

$$ \exists k \in \mathbb{N} \big((a_1+k,b_1+k)=(a_2,b_2)\big) \vee \big((a_1,b_1)=(a_2+k,b_2+k)\big) \iff (a_1,b_1) \equiv (a_2,b_2) $$

Definition der ganzen Zahlen über „kanonische Repräsentanten“:

$$\mathbb{Z} = \left\{ (a,0) | a \in \mathbb{N} \right\} \cup \left\{ (0,b) | b \in \mathbb{N} \right\} $$

