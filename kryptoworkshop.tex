\NeedsTeXFormat{LaTeX2e}
\documentclass[a4paper,10pt,bibtotoc,twoside,openright,pointlessnumbers,normalheadings,DIV=9
%,draft
]{scrbook}
\KOMAoptions{DIV=last}


\pagestyle{headings}
\usepackage{ngerman}
\usepackage[ngerman]{babel}
\usepackage[utf8]{inputenc}
\usepackage[T1]{fontenc}
\renewcommand{\sfdefault}{phv}
\renewcommand{\rmdefault}{phv}
\renewcommand{\ttdefault}{pcr}
\usepackage{graphicx}
\usepackage{verbatim}
\usepackage{tabularx}
\usepackage{subfigure}
\usepackage{url}
\usepackage{color}
\usepackage{amssymb}
\usepackage{amsmath}
\usepackage{amsthm}
\usepackage{setspace}
\usepackage{listings}
\lstset{language=Java,
  showstringspaces=false,
  frame=single,
  numbers=left,
  basicstyle=\ttfamily,
  numberstyle=\tiny}
\usepackage{pgfplots}
\usepackage{bussproofs}

% hier Namen etc. einsetzen
\newcommand{\fullname}{Stefan Schlott}
\newcommand{\email}{stefan@ploing.de}
\newcommand{\titel}{Notizen Kryptoworkshop Shackspace}
\newcommand{\jahr}{2012}

%color in tables
\usepackage{colortbl}
\definecolor{Gray}{rgb}{0.80784, 0.86667, 0.90196} %dunkelblau
\definecolor{Lightgray}{rgb}{0.9176, 0.95, 0.95686} %hellblau
\definecolor{Akzent}{rgb}{0.6627, 0.63529, 0.55294} %akzentfarbe
\setlength{\arrayrulewidth}{0.1pt}

\clubpenalty10000
\widowpenalty10000

\setlength{\parindent}{0pt}
\setlength{\parskip}{1.4ex plus 0.35ex minus 0.3ex}

% Tiefe, bis zu der Überschriften in das Inhaltsverzeichnis kommen
\setcounter{tocdepth}{3}

\pdfinfo{
  /Author (\fullname)
  /Title (\titel)
  /Producer     (pdfeTex 3.14159-1.30.6-2.2)
  /Keywords ()
}

\usepackage{hyperref}
\hypersetup{
pdftitle=\titel,
pdfauthor=\fullname,
pdfsubject={Workshop-Skript},
pdfproducer={pdfeTex 3.14159-1.30.6-2.2},
colorlinks=false,
pdfborder=0 0 0	% keine Box um die Links!
}

%Trennungsregeln
\hyphenation{Sil-ben-trenn-ung}

% Schönere Bullets bei Aufzählungen
% \renewcommand{\labelitemi}{$\bullet$}
% \renewcommand{\labelitemii}{$\circ$}
% \renewcommand{\labelitemiii}{$\cdot$}

% TeX-Tricks
% Dank UTF8 gehen auch Anführungszeichen toll:
% „“
% Tastenkombinationen:
% „ – Windows: AltGr+V, OS X: Alt+^
% “ – Windows: AltGr+B, OS X: Alt+2
% Geschützes Leerzeichen: ~


\begin{document}
\frontmatter

% Titelseite
\thispagestyle{empty}
\begin{addmargin*}[4mm]{-32mm}

\includegraphics[height=1.8cm]{images/logo_shack_brightbg}

\vspace{10cm}

\parbox{140mm}{\bfseries \huge \titel}\\[0.5em]
\end{addmargin*}


% Impressum
\clearpage
\thispagestyle{empty}
{ \small
  \flushleft
  „\titel“\\
  Fassung vom \today \\ \vfill

  \vspace{1cm}
  \copyright~\jahr~\fullname\\[0.5em]
% Falls keine Lizenz gewünscht wird bitte den folgenden Text entfernen.
% Die Lizenz erlaubt es zu nichtkommerziellen Zwecken die Arbeit zu
% vervielfältigen und Kopien zu machen. Dabei muss aber immer der Autor
% angegeben werden. Eine kommerzielle Verwertung ist für den Autor
% weiter möglich.
Dieses Werk ist unter der Creative Commons Attribution-NonCommercial-ShareAlike 3.0 Germany License lizensiert: \url{http://creativecommons.org/licenses/by-nc-sa/3.0/de/}
}


% ab hier Zeilenabstand 1,4 fach 10pt/14pt
\setstretch{1.4}

\tableofcontents

\mainmatter
\chapter{Klassische Chiffren}

\cite{Schneier1995}

\section{Caesar-Chiffre}

\section{rot13}

\section{rot n}

Monoalphabetische Chiffre

Unterscheidung Chiffre / Codierung: Chiffre besitzt Geheimnis; Codierung ist nur Darstellungsart.

\section{Modulo-Arithmetik}

\section{Angriff auf rot n}

Häufigkeitsanalyse

\section{Vigenère-Chiffre}

Polyalphabetische Chiffre

\section{Angriff auf Vigenere-Chiffre}

Zunächst Blocklänge bestimmen, anschließend für jede Gruppe eine separate Häufigkeitsanalye durchführen.

Blocklänge bestimmen: Durchprobieren, bis n-te Buchstaben charakteristische Verteilung ergeben

\section{Permutationen}

\section{Angriff auf Permutationen}

Blocklänge bestimmen: Markovketten-Modell der Sprache (Wahrscheinlichkeit, mit welcher ein Buchstabe auf den anderen folgt)
Durchprobieren, bis n-te Buchstaben dem Markov-Modell entsprechen

\section{Passwortstärke}

Längeres Passwort vs. größeres Alphabet: Länger ist besser


\section{Onetime-Pad}

Komplett zufälliger Bitstrom

Einzig wirklich sichere Quelle für Zufall: Quantenmechanik.

Nicht wiederverwenden (doppelt verwenden: Reduzierung auf Vigenere-Chiffre)


\chapter{Stromchiffren}

Stromchiffre = Pseudozufallszahlengenerator xor Klartext (bitweise)

Zufallszahlen generieren: Blockchiffre im Countermode:

Zähler mit (echt zufälligem) Schlüssel verschlüsseln --> ein Block Pseudozufalls-Bits

Auch hier: Zufallsstrom nicht wiederverwenden

Ausweg: Zufällig gewählten Initialwert für Counter, im Klartext mitübertragen

Vorsicht, Blockchiffre muß geeignet sein: Klartext ist ja bekannt (Counter), die Chiffre muß so konstruiert sein, daß selbst beim Vorliegen von Klar- und Chiffretext kein Rückschluß auf den verwendeten Schlüssel möglich ist.


\chapter{Zufallszahlen}

Für einen sicheren Generator darf die Folge nur rekonstruierbar sein, wenn der interne Zustand des Generators bekannt ist. Insbesondere darf aus $x_n$ nicht auf $x_{n-1}$ und $x_{n+1}$ geschlossen werden können.

\section{Linearer Kongruenzgenerator}

$$x_0 = s $$
$$x_{i+1} = (a x_i + b)\ mod\ n$$

$a$, $b$ und $n$ sind festgelegt; das „Geheimnis” für die Sicherheit steckt im Seed $s$. Um vernünftige Ergebnisse zu bekommen, sollten $a$ und $b$ teilerfremd zu $n$ sein. Offensichtlich ist der Generator unsicher: Der Folgewert kann direkt berechnet werden, der vorherige Wert läßt sich durch Lösen der linearen Gleichung bestimmen.

Üblicherweise benutzt man als Ausgabe jedoch nicht die gesamte Zahl, sondern nur einen Teil der Information (am besten nur ein einzelnes Bit, da man so möglichst wenig Information über den internen Zustand preis gibt). Das „rettet” das Verfahren dennoch nicht -- mit jeder Ausgabe kann eine weitere Gleichung eines linearen Gleichungssystems aufgestellt werden, welches schließlich zu lösen ist.

\section{Sicheres Verfahren auf Basis des Quadratwurzel-Problems}

Seien $p$ und $q$ (große) Primzahlen, $n=p q$.
$$x_0 = s $$
$$x_{i+1} = x_i^2\ mod\ n$$
$$r_i = x_i\ mod\ 2$$

Ausgabe $r_i$ ist pro Iteration ein einzelnes Bit (um über den internen Zustand nichts zu verraten). Die Sicherheit resultiert auf dem Quadratwurzelproblem, für welches keine effiziente Lösung bekannt ist.

\section{Chiffrierverfahren im Counter-Mode}

Wähle einen geheimen, rein zufälligen Schlüssel $k$. Verschlüssele mit diesem ein Zähler, der bei jedem Schritt inkrementiert wird:

$$x_i = E_k(i)$$

Problem bei Blockchiffren ist die Bijektivität: Da zu jedem Chiffretext exakt ein Klartext zugeordnet ist (und umgekehrt), kann über sehr lange Zeit beobachtet werden, welche Blöcke bereits aufgetreten sind; eine Wiederholung ist aufgrund der Bijektivität ausgeschlossen, was ein gewisses Bias ergibt. Blockchiffren ergeben somit zwar brauchbare, aber zumindest theoretisch angreifbare Zufallszahlengeneratoren.

In der Praxis ist das Verfahren sinnvoll sicher -- und weniger subtil als die Verwendung von Hashfunktionen (s.u.), simplicity wins :-)

\section{Verwendung von Hashfunktionen}

$$x_i = H(s+i) $$

Dies ergibt (sofern $H$ eine starke Hashfunktion ist) einen brauchbaren Zufallszahlengenerator. Eine Schwäche ist jedoch, daß mit dem Bekanntwerden des internen Zustands (z.B.~durch Einbruch in den entsprechenden Rechner) auch die Zufallszahlen der Vergangenheit berechnet werden können.

Verbesserter Ansatz:

$$x_0 = s$$
$$x_{i+1} = H(x_i + s)$$

Zu berücksichtigen ist, daß es hierbei zu verkürzten Perioden geben kann -- im Schnitt erreicht man eine Periode entsprechend der Quadratwurzel des Wertebereichs(??? ertes: Korrekt? Quelle?). Um Perioden zu vermeiden, kann $i$ noch einbezogen werden (so gibt es zwar Wiederholungen, aber Perioden werden durchbrochen):

$$x_{i+1} = H(x_i + s + i)$$

Um statistische Vorteile durch solche Wiederholungen zu vermeiden, sollte eine Hashfunktion mit entsprechend großem Wertebereich (SHA1 und besser) verwendet werden.



\chapter{Zahlentheorie}

Notation:

\begin{prooftree}
\AxiomC{$a \wedge b$}
\AxiomC{$a$}
\BinaryInfC{$b$}
\end{prooftree}

„Wenn alle Bedingungen oberhalb des Strichs erfüllt sind, gilt auch die Aussage unterhalb des Strichs“.


\section{Definition der natürlichen Zahlen}

\subsection{Axiome der natürlichen Zahlen}

„Die natürlichen Zahlen sind alle Werte von $x$, für die $n(x)$ gilt”:

$$\mathbb{N} = \left\{ x | n(x) \right\}$$

Notations-Definition: $x'$ ist der Nachfolger von $x$.


Die fünf Axiome der natürlichen Zahlen:

\begin{enumerate}
  \item $0$ ist eine natürliche Zahl:
  \begin{prooftree}
  \AxiomC{}
  \UnaryInfC{$n(0)$}
  \end{prooftree}

  \item Wenn $x$ eine natürliche Zahl ist, ist $x'$, der Nachfolger von $x$, auch eine natürliche Zahl:
  \begin{prooftree}
  \AxiomC{$n(x)$}
  \UnaryInfC{$n(x')$}
  \end{prooftree}

  \item Wenn der Nachfolger von $x$ gleich dem Nachfolger von $y$ ist, so ist $x$ gleich $y$:
  \begin{prooftree}
  \AxiomC{$x'=y'$}
  \UnaryInfC{$x=y$}
  \end{prooftree}

  \item Wenn $x$ eine natürliche Zahl ist, so ist ihr Nachfolger nie $0$:
  \begin{prooftree}
  \AxiomC{$n(x)$}
  \UnaryInfC{$x' \ne 0$}
  \end{prooftree}

  \item Wenn $x$ eine natürliche Zahl ist, so ist sie entweder $0$ oder sie besitzt einen Vorgänger $y$, welcher ebenfalls eine natürliche Zahl ist:
  \begin{prooftree}
  \AxiomC{$n(x)$}
  \UnaryInfC{$x=0 \vee (\exists y: y'=x \wedge n(y) )$}
  \end{prooftree}
\end{enumerate}


\subsection{Definition der Addition}

\begin{enumerate}
  \item $0$ ist das neutrale Element der Addition: $x+0=x$
  \item $x+y'=x'+y$ oder $x+y' = (x+y)'$
\end{enumerate}

Daraus läßt sich das Assoziativgesetz ableiten:

$$(a+b)+c = a+(b+c)$$


\subsection{Definition der Multiplikation}

\begin{enumerate}
  \item $x \cdot 0=0$
  \item $x \cdot y'=x+(x \cdot y)$
\end{enumerate}

Daraus läßt sich ebenfalls das Assoziativgesetz ableiten:

$$(a \cdot b) \cdot c = a \cdot (b \cdot c)$$

Konvention: Es gilt Punkt vor Strich; zwischen Variablennamen kann das Multiplikationszeichen weggelassen werden.


\section{Definition der ganzen Zahlen}

$$\mathbb{Z}' = \left\{ (a,b) | a \in \mathbb{N}, b \in \mathbb{N}, a+x=b \right\}$$

$\mathbb{Z}'$ definiert noch nicht die ganzen Zahlen, sondern eine Menge von Gleichungen; über $x$ wird noch nichts ausgesagt -- für die Definition steht ja nur $\mathbb{N}$ zur Verfügung. Bildung von Äquivalenzklassen der Paare $(a,b)$:

$$ \exists k \in \mathbb{N} \big((a_1+k,b_1+k)=(a_2,b_2)\big) \vee \big((a_1,b_1)=(a_2+k,b_2+k)\big) \iff (a_1,b_1) \equiv (a_2,b_2) $$

Definition der ganzen Zahlen über „kanonische Repräsentanten“:

$$\mathbb{Z} = \left\{ (a,0) | a \in \mathbb{N} \right\} \cup \left\{ (0,b) | b \in \mathbb{N} \right\} $$


\chapter{Gruppentheorie}

Sei $M$ eine Menge, sowie $*$ eine Funktion, welche zwei Elemente auf $M$ auf ein weiteres Element auf $M$ abbildet. $(M,*)$ nennen wir eine {\em Halbgruppe (Semigroup)}, wenn gilt:
\begin{enumerate}
  \item $*$ ist geschlossen, also $\forall e_1, e_2 \in M: e_1 * e_2 \in M$
  \item $*$ ist assoziativ, also $x*(y*z)=(x*y)*z$
\end{enumerate}

$(M,*,e)$ heißt {\em Monoid}, wenn zusätzlich gilt:

\begin{enumerate}
\setcounter{enumi}{2}
  \item $e$ ist neutrales Element, also $x*e = x = e*x$
\end{enumerate}

Ein Beispiel für einen Monoid ist die Operation $+$ über den Natürlichen Zahlen: $(\mathbb{N},+,0)$.

$(M,*,e)$ heißt {\em Gruppe}, wenn zusätzlich gilt:

\begin{enumerate}
\setcounter{enumi}{3}
  \item Für alle Elemente $x$ existiert ein inverses Element $y$:
  $$\forall x \in M\ \exists y \in M: x*y=e$$
\end{enumerate}


Es kann nur ein neutrales Element geben: Angenommen, wir hätten zwei unterschiedliche neutrale Elemente $e_1$ und $e_2$. Setze in Regel 3 $x=e_1$:
$$ e_1 * e_2 = e_1 = e_2 * e_1 $$
Setze in Regel 3 $x=e_2$:
$$ e_2 * e_1 = e_2 = e_1 * e_2 $$
Ineinandereingesetzt gilt damit auch:
$$ e_2 = e_1 * e_2 = e_1 $$
Dies ist ein Widerspruch zur Annahme.





\appendix
% hier Anhänge einbinden
%\input{chapters/sources}

\backmatter

\bibliographystyle{plaindin} % Nummern und alphabetisch sortiert
%\bibliographystyle{alphadin} % Buchstaben und sortiert
%\bibliographystyle{abbrvdin} % Nummern und abgekürzte Namen
%\bibliographystyle{unsrtdin} % Nummern und unsortiert
\bibliography{bibliography}

\end{document}
